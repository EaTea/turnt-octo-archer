\section{Introduction} \label{secIntro}

Software engineering is a field of engineering unlike any other engineering.
Unlike ``classical" engineering fields, which produce bridges, computers or
cars, software engineering products are intangible.
We claim that this intangibility results from software engineering being the
discipline of encoding a software developer's knowledge into different forms.\\
\\
But how does a software developer's knowledge affect the software engineering
process?
What are the measurable effects of having less knowledge of the problem we are
solving?
Or the measurable effects of having less knowledge of the resources we use?\\
\\
These questions interest us, since being able to say ``Since I know less about
this problem relative to my previous knowledge, I can infer that it will take me
this much longer to solve this problem" allows us to better estimate and control
our projects.
Furthermore, it informs us about learning and how knowledge carries over between
projects.
This might indirectly influence methods that improve software engineering
processes.\\
\\
In this paper, we will measure how lessening software developers' knowledge
changes their software development process.
We do so by analysing the development data of a group of students undertaking
different programming tasks multiple times.
We will examine and analyse multiple models that seem appropriate to
measuring effort and knowledge, and how they fit to the student group's development data,
as well as our own development data.\\
\\
The paper is structured as follows
\begin{itemize}
  \item section \ref{secMotivation} breaks our research goals
  into multiple sub-problems which are centred around a theme of software as
  an encoding of knowledge
  \item section \ref{secMethod} discusses our approach to our experiment and
  any experimental factors
  \item section \ref{secModels} discusses the models we use and justifies their
  usage
  \item section \ref{secResults} gives the results that we found, including how
  they fit to our data and how they fit to the group development data
  \item section \ref{secAnalysis} gives our analysis of the results, and answers
  the questions we define in our motivation
\end{itemize}
