\section{Models} \label{secModels}

In section \ref{secMethod} we outlined three different criteria that our models
needed to fulfill.

\begin{itemize}
  \item $a$: the amount of time taken to complete the solution with no knowledge,
  which should be the value the model takes on \AZ
  \item $c$: the amount of time taken to complete the solution with complete
  knowledge, which should be the value the model takes as the number of attempts
  goes to infinite --- this is the same as saying the model asymptotes at $c$
  \item $b$: the learning rate, which should determine the curvature of the
  model and control how fast the students learn
\end{itemize}

Furthermore, we inferred that $a > c > 0$ and $b > 0$, and this adds additional
implicit constraints on our models.
We present four models that were suggested by Woodings for usage.
Much of their motivation comes from an unpublished paper by Woodings. \FIXME
We apply each of the constraints that we have to ensure that they comply with
our criteria.\\
\\
For a model $m$, a function of $t$ (the number of iterations), our constraints, as equations, are
\begin{equation}
  t = 0 \implies m(t) = a
\end{equation}
\begin{equation}
  \lim_{t \to \infty} m(t) = c
\end{equation}
\begin{equation}
  b > 0
\end{equation}
\begin{equation}
  a > c > 0
\end{equation}

We present each of our models in subsection \ref{subsecModels} and justify why
they are valid (or invalid) models in subsection \ref{subsecAppropriateness}.

\subsection{Models} \label{subsecModels}

This following section is based on an unpublished paper by Woodings, which we
reproduce for the completeness of the paper.
It is a good motivation for each of the models we use, and it is clear, by
inspection, that each of the models will be able to fit the criterion.\\
\\
Our first model is based on the logarithmic Poisson distribution, which
deals with time-based data and incidents.
This hyperbolic model is ideal and has been used to analyse failure intensity.
This model is given by
\begin{equation} \label{modelOne}
  m(t) = \frac{a+bct}{bt+1}
\end{equation}

Our second model is a psychological model on learning, which is used by Kemerer
and Sherdii\FIXME referencing.
We can derive a similar equation to the one used in Humphries Personal Software
Process (PSP) \FIXME referencing, resulting in
\begin{equation} \label{modelTwo}
  m(t) = \frac{a-c}{(t+1)^{b}}+c
\end{equation}

Our third model is based on the idea that the improvement is proportional
smaller than
the effort we have previously invested --- that is, improvement is some function of the
previous iteration's effort.
This can be encapsulated by 
\[
  m(t+1) = k m(t)
\]
where $0 \leq k \leq 1$ is some constant.\\
\\
This then yields
\[
  m(t) = k^t 
\]
Adding the asymptote when $t \to \infty$ yields our third model
\begin{equation} \label{modelThree}
  m(t) = (a-c) e^{-bt}
\end{equation}

Our final model is a model presented by Woodings in \FIXME.
It is a simple polynomial, given as
\begin{equation} \label{modelFour}
  m(t) = a + bt + ct^2
\end{equation}
It might immediately be clear to the reader why this model is inappropriate, but
we will formally justify its exclusion in the subsection \ref{subsecAppropriateness}.

\subsection{Appropriateness of models} \label{subsecAppropriateness}
