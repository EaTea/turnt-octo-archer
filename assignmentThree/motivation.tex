\section{Research questions} \label{secMotivation}

Measuring ``knowledge" is a difficult process, since ``knowledge" is an
ill-defined characteristic, much like ``quality" or ``learning".
Answering questions like ``how much more do we know about A, compared to B?", or
``how much have we learnt?" are difficult tasks.
This is primarily because ``how much" implies some form of measurement which is
difficult with such a subjective characteristic.
Such questions are thus ineffective questions to answer when discussing the
relationship between knowledge and programming.\\
\\
We might instead use proxies and measurements from models to better this
relationship.
In order to do so, we must first redefine our research goals in terms of more
meaningful questions.
In this section, we will relate the questions in a broad and general manner.
Following this, section \ref{secMethod} will re-frame our questions in a more
objective manner.
We define seven questions that we will discuss and answer over the course of
this paper.\\
\\
Our first question asks
\begin{quote}
  To what extent does learning carry over between two software projects of similar
  specifications?
\end{quote}
To answer this question, we must measure how prior knowledge about problems affects
our ability to solve these problems.\\
\\
Our second question is similar to the first, and asks
\begin{quote}
  To what extent does learning carry over between two software projects using
  similar resources?
\end{quote}
To answer this question, we must measure how much effort is required to complete
two projects with identical specifications, but different resources.\\
\\
The third question we ask pertains to practice, and how it affects software
development.
\begin{quote}
  To what extent does practice enable us to better predict the effort required
  for software?
\end{quote}
This question discusses how good our measurements are, and any comments we can
make on the reliability of any results we have.\\
\\
Question four asks how learning differs as we practice on problems with similar
specifications and resources.
This is a measurement of ``second-order change", where we discuss how the
acquirement of knowledge affects changes in our learning process.\\
\\
Question five explores predictions on the amount of effort to produce software,
after all knowledge about the problem has been acquired.
Prior research suggests that the more time spent on designing and requirements during software
engineering the better the project outcome and success \cite{glass1998software,boehm1975some}.
If all the thinking has been performed, how long should it take to
finish a project?\\
\\
Our sixth question explores how practice improves code size and consistency, a nebulous
construct which we will define more concretely in discussing our method (Section
\ref{secMethod}).
We will attempt to answer how size changes between reiterations of the same
project, by comparing our own research data and practicing completion of a
problem.\\
\\
The final question we have discusses using models to fit our own data.
Which is the best model we have to use, and why is it appropriate?
Implicitly, we need to also discuss why other models might be inappropriate and
suggest reasons why this is the case.\\
\\
We have seven questions to answer, which are themed around the idea that
software is a process regarding knowledge.
Right now, however, they are immeasurable quantities, and although it is
desirable to answer them we cannot objectively perform measurements for them
as of yet.
Each one is still too nebulous and subjective to measure and discuss.
An objective framework is required to effectively answer these questions.
This is discussed in our next section on methodology.
