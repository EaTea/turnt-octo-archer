\section{Results} \label{secResults}

We present our results from each of our fits.
We used Wolfram Mathematica 9 to calibrate our models, and much of our analysis
assumes the correctness of Wolfram Mathematica 9's kernel and functions.\\
\\
This section of results is organised as follows.
Subsection \ref{subsecData} briefly gives some salient points about the data
itself, such as the mean time to complete each attempt and the standard
deviation.
Subsection \ref{subsecP1LA} shows the results of fitting our three models to
\PO\ using \LA.
Subsection \ref{subsecP1LB} shows the results of fitting our three moodels to
\PO\ using \LB.
Subsection \ref{subsecP2LA} shows the results of fitting our three models to
\PT\ using \LA.
Subsection \ref{subsecInd} shows individual results --- namely, my own and that
of a randomly selected student, to show how the models may be inappropriate for
an individual's design process.

\subsection{Analysing the data} \label{subsecData}

Some basic analyses that we performed on the data are presented below.
Although they do not contribute to answering our seven questions, we can make
some further comments and possible insights into the data and its usefulness.\\
\\
First, we begin by analysing the data for \PO\ when it was completed four
separate times in \LA, as shown in table \ref{tableP1LA}.

\begin{table}[ht!]
\centering
\begin{tabular}{|c|c|c|}
\hline
{\bf Attempt} &  {\bf Mean (minutes)} & {\bf Standard Deviation (minutes)} \\
\hline
\AZ & 172.857 & 91.6195 \\
\hline
\AO & 91.2143 & 69.0698 \\
\hline
\AT & 73.7857 & 49.6343 \\
\hline
\ATh & 40.2143 & 28.722 \\
\hline
\end{tabular}
\caption{Data from the analysis of \PO\ when completed with \LA.}
\label{tableP1LA}
\end{table}

This analyses suggests that the students began with a very wide range of abilities
and domain knowledge about the problem.
We can see what they began with needing almost 3 hours to finish the task, with a
standard deviation of more than 50\% of the mean.
As they continued to practice on the problem, the mean sharply decreased and the
standard deviation (approximately the spread of student times) began to contract.
However, the standard deviation as a percentage of the mean is still quite high.
This suggests that although the skill of the group as a whole was increasing, the
rates of learning, and the way that each student approached each iteration were quite
different across the entire group.\\
\\
Table \ref{tableP1LB} shows the analysis of the data for completing \PO\ in
\LB\ four times.

\begin{table}[ht!]
\centering
\begin{tabular}{|c|c|c|}
\hline
{\bf Attempt} &  {\bf Mean (minutes)} & {\bf Standard Deviation (minutes)} \\
\AZ & 102.643 & 40.3229 \\
\hline
\AO & 62.7143 & 38.0777 \\
\hline
\AT & 50.1429  & 30.1046 \\
\hline
\ATh & 43.5714 & 27.2134\\
\hline
\end{tabular}
\caption{Data from the analysis of \PO\ when completed with \LB.}
\label{tableP1LB}
\end{table}

We can immediately see that the students were much faster in their first attempt for
\PO\ in \LB.
We will compare their differences more in section \ref{secAnalysis}.
It is worth noting that the spread of times is much smaller and the mean times are lower.
However, we also note that subsequent attempts did not show as much improvement as in the
first set of data.
The mean of their final times was comparable to the mean time taken for the final attempt
of \PO\ in \LA, but the spread has dramatically decreased.
This suggests that the students' skills and knowledge were converging and suggests this
dataset was much more reliable.\\
\\
Table \ref{tableP2LA} shows the analysis of the data for completing \PT\ in
\LA\ four times.

\begin{table}[ht!]
\centering
\begin{tabular}{|c|c|c|}
\hline
{\bf Attempt} &  {\bf Mean (minutes)} & {\bf Standard Deviation (minutes)} \\
\hline
\AZ & 100.143 & 70.1415 \\
\hline
\AO & 54.2143 & 38.573 \\
\hline
\AT & 49.9286  & 38.8794 \\
\hline
\ATh & 39.1429 & 34.9436\\
\hline
\end{tabular}
\caption{Data from the analysis of \PT\ when completed with \LA.}
\label{tableP2LA}
\end{table}

This dataset suggests that the knowledge from the previous problem domain transferred
over to the new dataset quite well.
The average time required to complete the problem was lower, but the spread of times was much higher.
The spread also did not converge as readily or tightly as the previous dataset.
We note that the differences in times between each dataset were far more erratic.
This suggests that language and resource changes are easier to deal with than problem domain
changes.

\subsection{Problem one in language A} \label{subsecP1LA}

We first present our results for \PO\ in \LA.
Table \ref{table:P1LA:abc} shows our best fit parameters.
Table \ref{table:P1LA:abc:error} gives the standard errors for each of the
parameter estimates.
Table \ref{table:P1LA:abc:sumsquares} is the sum of squares of the residuals of
the model against the data, giving us a rough idea of how good the fit was.\\
\\
Before showing the data, a brief comment about the models.
This is the only dataset where models $m_1$ and $m_2$ were unable to give
sensible results.
This is also the dataset with the greatest change in times from \AZ\ to \ATh.
We conjecture without proof or evidence that models $m_1$ and $m_2$ are not
tolerant to substantial increases in knowledge (and implicitly greatly reduced
time taken in each attempt).

\begin{table}[ht!]
\centering
\begin{tabular}{|c|c|c|c|}
\hline
{\bf Model} &  $a$ & $b$ & $c$ \\
\hline
$m_1$ & 172.652 & 0.712 & -19.8246 \\
\hline
constrained $m_1$ & 172.931 & 281.107 & 64.8165 \\
\hline
$m_2$ & 172.607 & 0.638207 & -49.9154\\
\hline
constrained $m_2$ & 172.857 & $1.67721 \times 10^10$ & 65.0713 \\
\hline
$m_3$ & 172.322 & 0.794716 & 29.691 \\
\hline
\end{tabular}
\caption{Parameters of the best fits for each of the models for \PO\ in \LA.}
\label{table:P1LA:abc}
\end{table}

During the modelling process, $m_1$ and $m_2$ both found negative $c$ values to
give the best fit.
This is quite troubling --- it suggests that these models are both inappropriate
for use in modelling learning.
The constrained $m_1$ and $m_2$ models have the additional constraints given in
Equations \ref{conThree} and \ref{conFour}, that is
\[
  a > c > 0
\]
and
\[
  b > 0.
\]

Adding these constraints in did not result in a good fit --- Mathematica
returned that the fit did not converge and gave nonsensical values for $b$ (that
is the learning rate).
This suggests that both $m_1$ and $m_2$ are inappropriate for modelling repeated
iterations of a task.
$m_3$ finds a more suitable fit which suggests that it was more robust in
finding sensible fits.
A final comment is that the $a$ values are all very similar and have negligble
differences between them, suggesting that the $a$ values are relatively similar.
More information might be found by examining the confidence for each of the
parameter fits that Mathematica found.

\FIXME grapph here

\begin{table}[ht!]
\centering
\begin{tabular}{|c|c|c|c|}
\hline
{\bf Model} &  $a$ & $b$ & $c$ \\
\hline
$m_1$ & 4.51941 & 0.18701 & 19.8109 \\
\hline
constrained $m_1$ & 35.9144 & 53906.8 & 49.4857 \\
\hline
$m_2$ & 4.1785 & 0.236003 & 54.71152\\
\hline
constrained $m_2$ & N/A & N/A & N/A \\
\hline
$m_3$ & 6.34312 & 0.189644 & 13.1391 \\
\hline
\end{tabular}
\caption{Standard error on the parameters of the best fits for each of the models for \PO\ in \LA.
Note that when calculating the standard errors for the constrained $m_2$ the
  computer repetitively crashed and so these values are not included.}
\label{table:P1LA:abc:error}
\end{table}

We take the standard errors for each of the parameters from each of the fits that Mathematica yielded.
Notice that while the $a$ parameters have low errors, the errors for $c$ and $b$ are
much greater (indeed, they are significant percentages of the parameters $c$ and
$b$ --- as an example, giving a confidence interval of approximately $\pm
23.78\%$ for the $b$-value of $m_3$).
Even more interesting are the standard errors for $b$ in the constrained $m_1$
and $m_2$.
This is a ridiculous
figure in the constrained $m_1$ and it unfortunately crashed Mathematica (and my
computer) when calculating the errors for the constrained $m_2$.
This suggests that these models did not fit very well to the data given, and
that although our value for $a$ is relatively accurate our value for $c$ is
quite inaccurate.

\begin{table}[ht!]
\centering
\begin{tabular}{|c|c|}
\hline
{\bf Model} & Sum of squares \\
\hline
$m_1$ & 20.4672649169\\
\hline
constrained $m_1$ & 1290.3821781171\\
\hline
$m_2$ & 17.5223504018\\
\hline
constrained $m_2$ & 1302.9796653729\\
\hline
$m_3$ & 40.5502881745 \\
\hline
\end{tabular}
\caption{Sum of squares of the residuals for the best fits for each of the models for \PO\ in \LA.}
\label{table:P1LA:abc:sumsquares}
\end{table}

It is most interesting to note that although the constrained $m_1$ and $m_2$ did
not fit well (and hence we ought to disregard them),
the unconstrained models $m_1$ and $m_2$ had very low sums of
squares compared to $m_3$ --- this suggests that perhaps they could fit the data
very well, but not in a sensible manner.
This seems to support the conjecture that they are not tolerant to such high
learning rates and will give nonsensical results.

\subsection{Problem one in language B} \label{subsecP1LB}

We now present our results for \PO\ in \LB.
Table \ref{table:P1LB:abc} shows our best fit parameters.
Table \ref{table:P1LB:abc:error} gives the standard errors for each of the
parameter estimates.
Table \ref{table:P1LB:abc:sumsquares} is the sum of squares of the residuals of
the model against the data, giving us a rough idea of how good the fit was.

\begin{table}[ht!]
\centering
\begin{tabular}{|c|c|c|c|}
\hline
{\bf Model} &  $a$ & $b$ & $c$ \\
\hline
$m_1$ & 103.736 & 1.05646 & 25.0618 \\
\hline
$m_2$ & 102.633 & 1.05789 & 25.9886\\
\hline
$m_3$ & 102.57 & 1.03095 & 41.3707 \\
\hline
\end{tabular}
\caption{Parameters of the best fits for each of the models for \PO\ in \LB.}
\label{table:P1LB:abc}
\end{table}

Notice again that the $a$ values are relatively similar.
This time, the $b$ values are also quite close, and the $c$ values are the only
real differences in the models.
Each of these models gave sensible results, which suggests that since the
learning rate between each iteration seemed lower compared attempts in \PO\ in
\LB, the model is able to give useful and meaningful results.

\FIXME grapph here

\begin{table}[ht!]
\centering
\begin{tabular}{|c|c|c|c|}
\hline
{\bf Model} &  $a$ & $b$ & $c$ \\
\hline
$m_1$ & 0.200046 & 0.0268934 & 0.635585 \\
\hline
$m_2$ & 0.224989 & 0.0312964 & 1.17134\\
\hline
$m_3$ & 1.24103 & 0.0954955 & 1.80347 \\
\hline
\end{tabular}
\caption{Standard errors for the parameters of the best fits for each of the models for \PO\ in \LB.}
\label{table:P1LB:abc:error}
\end{table}

The errors in Table \ref{table:P1LB:abc:error} are miniscule in comparison to
Table \ref{table:P1LA:abc:error}.
Notice that $m_1$ had the smallest errors, whilst $m_3$ had consistently higher
errors than $m_1$ or $m_2$.
This suggests that $m_3$ might be more tolerant to high learning rates, but
gives a poorer fit overall.

\begin{table}[ht!]
\centering
\begin{tabular}{|c|c|}
\hline
{\bf Model} & Sum of squares \\
\hline
$m_1$ & 0.0400487707\\
\hline
$m_2$ & 0.0507152055\\
\hline
$m_3$ & 1.5429195761\\
\hline
\end{tabular}
\caption{Sum of squares of the residuals for the best fits for each of the models for \PO\ in \LB.}
\label{table:P1LB:abc:sumsquares}
\end{table}

The results of the standard errors in Table \ref{table:P1LB:abc:error} are
confirmed in Table \ref{table:P1LB:abc:sumsquares} (the sum of squares of the
residuals).
Although alll models have very low sum of squares, $m_3$ has a greater sum of
squares of the residuals than $m_1$ and $m_2$.
Furthermore, $m_1$ again had the lowest sum of squares (and as shown in Table
\ref{table:P1LB:anc:error} the lowest errors for the parameters) which suggests
that here, it was the best fit.

\subsection{Problem two in language A} \label{subsecP2LA}

We now present our results for \PO\ in \LB.
Table \ref{table:P2LA:abc} shows our best fit parameters.
Table \ref{table:P2LA:abc:error} gives the standard errors for each of the
parameter estimates.
Table \ref{table:P2LA:abc:sumsquares} is the sum of squares of the residuals of
the model against the data, giving us a rough idea of how good the fit was.

\begin{table}[ht!]
\centering
\begin{tabular}{|c|c|c|c|}
\hline
{\bf Model} &  $a$ & $b$ & $c$ \\
\hline
$m_1$ & 100.079 & 1.78221 & 30.7798 \\
\hline
$m_2$ & 100 & 1.62075 & 34.7768\\
\hline
$m_3$ & 99.9429 & 1.39094 & 41.4713 \\
\hline
\end{tabular}
\caption{Parameters of the best fits for each of the models for \PT\ in \LA.}
\label{table:P2LA:abc}
\end{table}

Again, as in Table \ref{table:P1LA:abc} and \ref{table:P1LB:abc} we note that
the $a$ values as shown in Table \ref{table:P2LA:abc} are very similar.
The $b$ values show much more difference than in Table \ref{table:P1LB:abc}, and
the $c$ values show a reasonable variation (this is by inspection only, and
without justification through evidence).

\FIXME grapph here

\begin{table}[ht!]
\centering
\begin{tabular}{|c|c|c|c|}
\hline
{\bf Model} &  $a$ & $b$ & $c$ \\
\hline
$m_1$ & 4.94326 & 1.25419 & 11.5227 \\
\hline
$m_2$ & 5.18028 & 0.883591 & 13.2946 \\
\hline
$m_3$ & 6.08996 & 0.662606 & 6.49302 \\
\hline
\end{tabular}
\caption{Standard errors for the parameters of the best fits for each of the models for \PT\ in \LA.}
\label{table:P2LA:abc:error}
\end{table}

Table \ref{table:P2LA:abc:error} shows some interesting results.
First, we see that the $a$ values have similar errors that are quite small,
whilst the $b$ values have significant errors, suggesting confidence intervals
that can be as high as $\pm 70.372\%$ (for the $b$ value of model $m_1$).
Secondly, we note that the $c$ values had high errors in all models except $m_3$
--- indeed, $m_3$ had extremely low errors for the $b$ and $c$ values.
We ought to note that in Table \ref{tableP2LA} the differences between the means
of each attempt are quite extreme and perhaps are not as amenable to analysis.

\begin{table}[ht!]
\centering
\begin{tabular}{|c|c|}
\hline
{\bf Model} & Sum of squares \\
\hline
$m_1$ & 24.439796138\\
\hline
$m_2$ & 26.8556022611\\
\hline
$m_3$ & 37.098573229\\
\hline
\end{tabular}
\caption{Sum of squares of the residuals for the best fits for each of the models for \PT\ in \LA.}
\label{table:P2LA:abc:error}
\end{table}

The sum of squares of residuals as shown in Table \ref{table:P2LA:abc:error}
supports our suggestions that this dataset was not as amenable to analysis.
Again we see that $m_1$ had the lowest sum of squares, which suggests it was the
best fit.\\
\\
From our data, we would suggest that $m_1$ would give the best fits from our analysis of \PT\
in \LA\ and \PO\ in \LB.
However, the useless results it gives in analysing the dataset for \PO\ in \LA\
suggests that it is not tolerant to extremely high learning rates and should be
used with care.
We also note that $m_3$ will consistently give good and useful results but will
not fit as well as $m_1$, though it is a good alternative when the $m_1$ gives
bad results.
Since we have so few data points, it is a very rudimentary and early analysis
and we note that our results are still not very reliable.
We will next try using our models $m_1$, $m_2$ and $m_3$ to analyse my data and
a random student's data.

\subsection{Individual Data}

\begin{table}
\centering
\begin{tabular}{|c|c|c|c|c|}
\hline
Attempt & \AZ & \AO & \AT & \ATh\\
\hline
Lines of code & 224 & 114 & 98 & 98 \\
\hline
\end{tabular}
\caption{My own individual data for lines of code.}
\label{table:loc}
\end{table}

\begin{figure}
\centering
\FIXME
\caption{The lines of code as a line graph --- notice that it may also be
  amenable to a similar analysis as the one we performed in the previous
    subsections.}
\label{figure:loc}
\end{figure}


