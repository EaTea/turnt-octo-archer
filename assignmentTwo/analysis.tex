\section{Analysis} \ref{analysis}

Here, we analyse our modelling method, the different attributes of the strategies, and which
strategies we would use overall.

\subsection{Modelling Method}

This project was really quite interesting, in the sense that it was entirely based around
simulation.
The challenges that I faced in translating this from a purely mathematical to a process-based
exercise were significant.\\
\\
I deigned it a good idea to start coding and hack out a simulation framework.
This really blocked me from doing other simulations or parts of the project, and though it was a
worthwhile and interesting task it was very fiddly and full of problems.\\
\\
As mentioned before, it resulted in negative results for absolute scales, which is a rather silly
thing to have!
It also meant that at some point, I had to put faith in my models and {\em hope} that I had built it
correctly.
This is true of all simulations and models, but I relied on more unproven and untested frameworks
than my peers.
Although my results seem sensible and normal, I think this was a risky approach to simulation and,
given the time constraints for this assignment, was not a wise decision.\\
\\
I thought my modelling methodology was strong, however.
Theory-wise, I felt that the assumptions I made were necessary to build the model, but at least
acknowledging them made it understandable as to why an assumption was made.
Furthermore, I think that at least acknowledging that it is a mistake allows us to open it up for
further work, and I would claim that my framework is close to being able to remove some of these
assumptions.\\
\\
I believe that the specifications of the assignment were open for a reason, as it is somewhat a
research project.
This openness meant that I could afford to make my own judgements about situations in ways I
believed a project manner would have.
These judgements translated into assumptions about how the defect queue should be ordered or when a
specific resource reallocation should occur.\\
\\
Nevertheless, we reach a limitation of my modelling approach --- I encoded these assumptions without
a review of literature or looking at previous work.
I felt I was too time-pressed to review literature and back up my judgements and assumptions.
I feel that the lack of empirical evidence in some of my assumptions detracted from my work.

\subsection{Defect Resolution Strategies}

We will analyse our strategies and the metrics we evaluated them with against the following criteria.
\begin{itemize}
	\item is the system reliable?
	\item does the system satisfy the client?
\end{itemize}

The first of these questions

\subsection{Future Work}

\FIXME explicitly outline any future work that could be undertaken
