\section{Analysis} \label{analysis}

Here, we analyse our modelling method, the different attributes of the strategies, and which
strategies we would use overall.

\subsection{Modelling Method}

This project was really quite interesting, in the sense that it was entirely based around
simulation.
The challenges that I faced in translating this from a purely mathematical to a process-based
exercise were significant.\\
\\
I deigned it a good idea to start coding and hack out a simulation framework.
This really blocked me from doing other simulations or parts of the project, and though it was a
worthwhile and interesting task it was very fiddly and full of problems.\\
\\
As mentioned before, it resulted in negative results for absolute scales, which is a rather silly
thing to have!
It also meant that at some point, I had to put faith in my models and {\em hope} that I had built it
correctly.
This is true of all simulations and models, but I relied on more unproven and untested frameworks
than my peers.
Although my results seem sensible and normal, I think this was a risky approach to simulation and,
given the time constraints for this assignment, was not a wise decision.\\
\\
I thought my modelling methodology was strong, however.
Theory-wise, I felt that the assumptions I made were necessary to build the model, but at least
acknowledging them made it understandable as to why an assumption was made.
Furthermore, I think that at least acknowledging that it is a mistake allows us to open it up for
further work, and I would claim that my framework is close to being able to remove some of these
assumptions.\\
\\
I believe that the specifications of the assignment were open for a reason, as it is somewhat a
research project.
This openness meant that I could afford to make my own judgements about situations in ways I
believed a project manner would have.
These judgements translated into assumptions about how the defect queue should be ordered or when a
specific resource reallocation should occur.\\
\\
Nevertheless, we reach a limitation of my modelling approach --- I encoded these assumptions without
a review of literature or looking at previous work.
I felt I was too time-pressed to review literature and back up my judgements and assumptions.
I feel that the lack of empirical evidence in some of my assumptions detracted
from my work.\\
\\
Other things that I think would have helped would be to do more comparisons with
my data; I would like to have seen how different variants of a strategy matched
up against each other.
Unfortunately, I did not have the foresight and resources to do this!
I would reserve this for future work to see how tweaking strategies improves or
degrades strategy performance.

\subsection{Defect Resolution Strategies}

We will analyse our strategies and the metrics we evaluated them with against the following criteria.
\begin{itemize}
	\item is the system reliable?
	\item does the system satisfy the client?
\end{itemize}

The first of these questions revolves around how many defects we left in the
system by the time it finished.
An example metric we could use here would be mean-time-to-failure.
We could estimate the number of remaining bugs, and divide it by the
number of weeks that had elapsed.\\
\\
The idea of estimating system reliability using this metric
  is less interesting in the sense that all the
strategies (excluding FIFO with external influences) ended up with very similar
amounts of
defects left by the time the strategy was finished.
Furthermore, we note that all of the strategies had begun to stabilise (around 5
defects left and no new data being read in), suggesting that all the systems
delivered a system of similar reliability.\\
\\
In addition, most of the systems managed to account for almost all major
defects.
This suggests that the strategies were overall effective in capturing and
constructing a reliable system.
I would contend that there is actually not very much interesting insights to
talk about here.
Perhaps it is an artifact of my methodology, but because all
my metrics had the same terminating condition and that terminating condition
revolved around a reasonably accurate extrapolation of the number of remaining
defects, they all delivered the same level of system reliability.\\
\\
More interesting is the idea ``does our system satisfy the client?"
Here, we have a lot of richer and more worthwhile discussions.\\
\\
{\em Who} is the client?
An external entity?
A higher level of the organisational hierarchy?
Are they an ``internal" or ``external" stakeholder?
(e.g. are we developing an internal subsystem or product for the large company
 we are a part of, or are our services being contracted out as a smaller software house?)
Furthermore, if there are conflicts of interest, is there a strategy that
somewhat satisfies them all?\\
\\
Let us suppose that our client is an external entity.
Clients will use external pressure to attempt to have a subsystem be usable
faster.
Our analysis suggests that when under such external pressure, the best strategy
to adopt is to fix the major defects first.
Furthermore, the specific variant of the strategy that seems to be quite
effective would be to value major defects first, whilst using the size of the
defects queue to judge whether resource reallocation is necessary.
This is somewhat surprising to me, but it is worth noting that the resource
allocation here seems to be a more powerful strategy overall.\\
\\
Let us suppose thast our ``client" is in fact an internal entity.
We might say that they are our boss or team leader.
Perhaps we are the software project manager for a team in a large technology
company and we are building the latest product to be marketed.
In this case, to satisfy the client is to actually satisfy internal forces.
If internal pressure is being applied, it is interesting to note that the FIFO
strategy might be optimal here.
We refer to Table \ref{summary_in} to back up our claim, as FIFO seems to
actually do well in at least two of the three internal metrics.\\
\\
What happens if there is a conflict of interest between internal and external
stakeholders?
We tentatively suggest that the best approach is to use major-fixes-first,
   whilst using resoure reallocation based on the size of the defects queue.
The qualification ``tentatively" is required since we have a flawed measurement
for the external stakeholders measurement.\\
\\
We have explored different cases for strategies to choose, why we choose them
and which situations they are appropriate for.
Overall, prioritising client (external) satisfaction (that is, user
    satisfaction) seems more important than organisation requirements, due to
the long term benefits and the ethical obligations that I have as a software
developer.
Thus, I would personally employ a strategy that fixed the major defects first,
  and would reassign resources based on the size of the defect queue.

\subsection{Future Work}

Here, we discuss future work that this project has generated.
Firstly, we need to address the weaknesses in simulation framework.
It is clear that my framework is insufficient as is --- it is flawed in some
ways.
Fixing this would be the first step to extending my work.\\
\\
Secondly, extending the simulation framework would be interesting.
There are some assumptions, particularly involving defects not being
reintroduced, the unreliability of estimates regarding severity and time and the
parallelisation of the defect finding-fixing process that I would have liked to
have generalised within the strategy.\\
\\
Instead, I had to make the assumption that these were fixed or irrelevant, which
I believed weakened the model.
I was not able to test my model against a broader range of test environments as
a result.
Whether or not it is a good idea to make the model account for all assumptions
is perhaps another matter, since there are quite possibly diminishing returns on
addressing some of these open questions and the assumptions that we've made.\\
\\
Finally, I discuss a proposed automated learner for strategy classification.

\subsubsection{Automated Learner for Strategy Classification}
I constructed a small, currently ineffectual automated learner.
It would be conceivable that with this high number of metrics, a machine learner
could use these metrics to devise better, improved strategies.\\
\\
A suggested approach would be to evolve a strategy based on a subset of good,
accurate metrics.
Algorithms such as neural networks, evolutionary algorithms or particle swarm
optimisations come to mind.\\
\\
Essentially, with test data from many projects, it would be possible to try a
bunch of strategies, use a fitness function constructed from our metrics and
evolve a strategy that was suitable to maximising those metrics' scores.\\
\\
It would involve the following
\begin{itemize}
  \item transforming a strategy into a space suitable for machine learning ---
  finding an appropriate mapping, as well as specifically which attributes to
  map is an important question
  \item using an appropriate learning algorithm is also tantamount, and the ones
  I suggested are without basis or research
  \item use an appropriate set of metrics to construct a fitness function that
  accurately reflects what a project would want out of a strategy
  \item be able to generate possible strategies for arbitrary sets of data based
  on learned data and values from previous or strategies for learning on a new
  data set
\end{itemize}

I am reminded of the advances used to improve the COCOMO model, and the machine
learning acorss many projects that the project used.
This future work follows in the same vein but we leave it as an open question of
whether this would be an appropriate or rewarding way to find a strategy for
building software.
