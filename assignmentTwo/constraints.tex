\section{Constraints --- TODO by 30 August 2013}

\FIXME Fluff text

Modelling is the field of formalising assumptions about the world in logical and mathematical terms.
The formality of such models allows us to test, validate and analyse our model using objective, rather than subjective observations.
However, the more assumptions we make, the more specific our model is.
Perhaps it even makes our model less useful, because it is too specific.
Conversely, the less assumptions we make, the more general our model is.
But implicitly, it is more difficult to model (as there are more variables to capture).\\
\\
As we are building a (formal) model for defect detection and fixing, it is implicit that we too will make assumptions that we encode into our model.
It is thus worthwhile to discuss our observations and assumptions.
More importantly, we ought to justify why assumptions could be made and what their impact on the results are.\\
\\
This chapter begins by describing our problem, in terms of observations and open questions.
We then lay down our assumptions to answer the open questions, as well as justifications for {\em why} these assumptions were made.
We also discuss how this weakens our result, and its impact on finding the ``best" strategy.
Finally, we headline whether we will attempt to challenge our model later by removing or changing a specific assumption.

\subsection{Observations}

\FIXME Lay out the problem and put down as many observations and open questions
as possible

\subsection{Assumptions}

\FIXME Say how many assumptions and justify each one
