\section{Introduction}

Why Model?\\
\\
Epstein poses this as a rhetorical question, with an implicit retort that “You
are a modeller”. His claim is that to predict, analyse or understand a real life
situation is to model it. Furthermore, to construct valid models --- models that
are verifiable and “valid” --- is an artifact of the modelling process. Epstein
motivates this idea with respect to social or historical phenomena (such as
    wars, migrations and diseases) \cite{epstein2008model}.\\
\\
We can equally apply models to the software development process.
COCOMO \cite{boehm1984software} and Putnam's SLIM \cite{putnam1978general} are examples of models of the software development process.
They focus on the idea of costing the software development process.
In particular, modelling allows us to evolve strategies for new projects by analysing previous projects.
That is, insights gathered from previously modelled projects can yield more reliable, accurate insights about a new project.\\
\\
Within this paper, we are especially interested in the defect detection and amendment.
We define a defect as
\begin{quote}
  a misunderstanding or mistranslation of an idea into code that results a
  failure experienced by a user of a software system
\end{quote}

It is clear that from a client perspective, defects negatively affect software that clients use. 
Indeed, how timely the detection and fixes of defects occur is an important part of software development.
\cite{gaugingStakeholderPerc} shows that client perception is negatively
influenced by late fixes and lack of timeliness.
It relates back to the ``quality" of the software development service, a soft and hard-to-define
characteristic that governs the outcomes of many projects.
Furthermore, \cite{green2005impacts} shows that there is a need to measure
processes in order to improve quality and the perception of ``goodness" in a
service.\\
\\
This paper aims to answer the question: what is the ``best” way to find and fix defects in a project?
Implicitly, we must also define what the ``best" is.
Our contribution to answering this question is fourfold.
We will
\begin{enumerate}
  \item construct a basic model of the defect detection and reparation process
  \item use previous defect detection data within our model to measure the effectiveness of different defect testing and fixing strategies
  \item analyse the performance of the strategies we have used, as well as challenging the models we have constructed (and implicitly, of the effectiveness of the strategies we have employed)
	\item analyse what our metrics mean, what it means to perform ``well" according to a metric and
answer what the ``best" strategy is
\end{enumerate}
