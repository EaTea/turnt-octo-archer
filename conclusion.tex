\chapter{Conclusion}

We have examined a multitude of different techniques for defect detection and prevention, focusing
on those that have received much attention within the last decade.
In particular, we have focused on code reviews, and compared them to other defect
detection/prevention techniques.
This is because the of wide variety of verification techniques that now exist in software
engineering --- simply focusing on specialising Fagan's techniques even more, or nuances of code
reviews, are ineffective.\\
\\
Instead, we have given a rudimentary analysis of the current state of the art of code reviews ---
how they relate to other techniques, how they are affected, and what the benefits and weaknesses
that they still carry relative to those other techniques are.
The experiments we have devised are a means to fill the gaps in knowledge that we have found, and in
doing so, propose an empirical way forward to constructing a well-formed defect prevention and
detection pipeline that is supported by academic and empirical evidence.
In doing so, this work would hope to advance the state of the art by attempting to construct a
basis for a general defect detection/prevention framework that centres around the concept of review
as its main technique.
