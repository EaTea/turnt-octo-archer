\chapter{What are defects?} \label{chapter:defects}
Defects are common-place occurrences in software development.
It is almost inevitable that software has defects and flaws, which causes software errors and
faults.\\
\\
But what are defects?
How does literature define defects?
And what are some of the problems that arise from seemingly small defects,
and their major after-effects?\\
\\
This chapter is arranged as follows:
\begin{itemize}
	\item Section \ref{sec:defects:defect} discusses what defects are and why they occur
	\item Section \ref{sec:defect:lifetime} discusses the lifetime of a defect
	\item Section \ref{sec:defects:defEffects} explores some of the problems
  that have resulted from seemingly small defects
\end{itemize}

\section{What is a defect?} \label{sec:defects:defect}

Let us consider the example of an ATM, used to withdraw money from, or check the balance of a bank
account.
How can we define the defects in a system?
To effectively do so, we must first define the idea of {\em customer expectations}.
\begin{quote}
	Customer expectations are the implicit or explicit conceptions of abilities and properties a
	system should have.
	These expectations come in two forms:
	\begin{itemize}
		\item {\em functional}: related to ``doing" or performing an action or task
		\item {\em non-functional}: related to a quality or characteristic of the design.
	\end{itemize}
\end{quote}

An example of such expectations in our bank account example is easily seen in the following:
\begin{itemize}
	\item the customer expects that the ATM can withdraw money from a bank account, which is a
		functional expectation
	\item the customer expects the ATM to encrypt all networked communications, in a secure manner ---
		this is a non-functional expectation
\end{itemize}

This is very similar to the idea of non-functional and functional requirements.
The only difference is semantic, with regards to ``requirements" compared to ``expectations".
``Expectation" highlights the importance of the user, and the (sometimes implicit) conception of
how a system ought to work that ``requirement" does not capture.
More information about functional and non-functional defects can be found in the comprehensive
text by Sommerville \cite{sommerville1989software}.\\
\\
We can then define a {\em defect} with respect to this idea, as
\begin{quote}
	A defect is a failure to meet customer expectations.
\end{quote}

\section{Lifetime of a defect} \label{sec:defect:lifetime}

Defects have been shown to have many evolutions.
Their effect upon a system can be shown to change, depending upon the stage of development and
usage of the system.
Musa et. al \cite{musa1987software} discuss the evolution of defects, from their origins within
code to their effects upon users in a software system.
The evolution stages Musa et. al describe are
\begin{itemize}
	\item a {\em fault} in the code constitutes a mistake in a piece of code as an embodiment of
		knowledge
	\item an {\em error} in system is a bad state that resulted from a fault
	\item a {\em failure} is caused by an {\em error} in a system
\end{itemize}

When a failure occurs it is difficult to trace where the original fault that caused it evolved from.
Machine testing on a finished product is simply not an effective debugging process.
Indeed, it can take as much as 100 times more person hours and resources to effectively track down
where the fault came from, according to Boehm et. al \cite{boehm2005foundations}.
This is supported by similar claims made by Soni \cite{soni2006defect} which are somewhat more
modern --- within it, he found that fixing early saved a factor of 3 times more than fixing defects
in the production stage.
Li et. al \cite{li2010transition} report similar results when changing from a plan-based software
life-cycle approach to an iterative Scrum-based approach, in terms of constant defect fixing and a
more efficient software development process.\\
\\
I believe that this trend, outlined by Boehm and supported by further research is a sensible one.
Woodings \cite{terryLecture4220} notes a further analogy about a dragon.
He notes that it is far easier to slay a dragon when it is a baby, than to try and slay the fully
grown dragon.
This metaphor highlights the importance of fixing a defect early --- especially at the stage when it
is a fault, rather than a failure.\\
\\
Fixing a fault, rather than a failure or error, requires detecting defects early.
Some schools of thought ascribe to testing early, and testing early and often (see Test Driven Development, in
subsection \ref{sec:otherdets:TDD}).
However, some software characteristics are not easily amenable to machine testing --- how do we test
``security" or ``extensibility"?
What happens if there is not enough code to test?
This motivates the need to find techniques that can be deployed within the implementation stages
when the amount of available code for testing is too small to be amenable to machine verification,
or what we need to verify does not lend itself easily to testing.

\section{The effects of defects} \label{sec:defects:defEffects}

To highlight the effects of seemingly simple errors, I will use three examples of real-world
software bugs that caused a major problem in the resultant systems.
In doing so, it will become clear how small problems that are not caught early can be catastrophic,
and furthermore how the smallest differences in how knowledge is encoded into software can
drastically affect system performance.
By exploring the nature of defects, why they occur and how they interplay with representations of
knowledge, we motivate the need for software to be reviewed by humans, and not just machine-tested
or analysed.\\
\\
The first example is related to video games --- namely, the popular video game ``EVE Online".
EVE is a massively multi-player role-playing game (MMORPG), with a large user base and a virtual
economy \cite{EVEWebsite}.
In December 2007, EVE Online was patched with the ``Trinity" patch, which deleted the ``boot.ini"
file inside the EVE Online software directory.
However, this was coded incorrectly, and by assuming that the ``boot.ini" file was being deleted
within EVE Online's directory, the authors of the patch instead deleted the system ``boot.ini" file
which was essential for Windows XP to run \cite{EVEUnboot}.\\
\\
This is a small fault in the code --- the author assumed that one file was being deleted and it
happened to have the same name as another file.
That the name of the file in question is the object used to boot the entire operating system
escalated the issue to severely compromise the systems of hundreds, or possibly thousands of
users.\\
\\
The second example involves security, and a small utility called ``Cryptocat".
Cryptocat offers secure instant messaging services that are not tied to services of any sort.
It is popular amongst activist groups for the purposes of communication in a secure manner
\cite{CryptocatWebsite}.
Unfortunately, an old patch in Cryptocat's random number generator, which ensured its keys were
secure, instead made all the keys used to encrypt the communications predictable.
The patch was simply an off-by-one error, but it subsequently compromised the integrity of thousands
of users communications \cite{CryptoShit}.\\
\\
The final example revolves around NASA's Mariner 1 --- an ill-fated voyage that ended with the
Mariner 1 being destroyed during operation.
This was caused by a mathematical formula being misread, which resulted in an incorrect
encoding into software.
As a result, the Mariner 1 destroyed itself, simply because the transcriber had not noticed there
was an overbar in one of the equations \cite{MarinerBoom}.\\
\\
These defects resulted in a fault in the code, and the fault was so small that it is hard to believe
it caused such a huge problem.
These are faults that come from a human attempting to transcribe their knowledge into a software
system.
Some of these bugs might have been amenable to machine testing, but many of them are not easily
tested.
How do you test for randomness?
What if you misread a formula and your test for the formula was wrong, in addition to the code?\\
\\
Verifying that knowledge is transcribed correctly is thus not a problem that is easily automated and
solved by computers.
This motivates the ideas of review, and the need to look beyond testing via a machine.
