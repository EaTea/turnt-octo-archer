\chapter{Introduction}

\begin{quote}
Software inspection is a method of static testing to verify that software meets
its requirements.
$\ldots$
Users of the method report very significant improvements in quality that are
accompanied by lower costs and greatly reduced maintenance efforts.
Excellent results have been obtained by small and large organizations in all
aspects of new development as well as in maintenance.
$\ldots$ developers who participate in the inspection of their own product
create fewer defects in future work.
\end{quote}

Michael Fagan \cite{AdvancesInSoftwareInspection} makes these bold claims of a
software verification method to trump all others in his seminal review of {\em
  software inspections}, or {\em code reviews}.
His review, {\it Advances in Software Inspections}, provides a broad overview of
literature discussing code reviews.
It argues that code reviews will, in no uncertain terms, improve the quality,
   cost, schedule and maintenance of software projects.\\
\\
Fagan's work, although far-reaching and (in my opinion) influential, is also
dated.
A work from 1986 can hardly be said to be ``modern", and its impact and
relatedness to today's software development industry has been greatly reduced.
New defect detection techniques and ideas, such as extreme programming,
test-driven development, red-green refactoring and automated static
inspection have greatly changed the defect detection and prevention
environment.\\
\\
Within this paper, we aim to update Fagan's seminal work with our own
review of software inspections.
We will compare software inspections to new methods of defect prevention and
detection, and argue for the relevance and usefulness of software inspections.
This comparison is necessary because the number and variety of defect detection and prevention
techniques as grown dramatically, and there are a number of key software verification mechanisms
that now exist.
We will aim to highlight the unanswered questions that revolve around code
reviews and propose a series of experiments that could be performed to
contribute to answering these questions.\\
\\
This paper is arranged as follows:
\begin{itemize}
  \item Chapter \ref{chapter:defects} defines defects as in literature, gives
  our own definition 
  \item Chapter \ref{chap:otherdets} discusses a range of newer defect detection
  and prevention techniques, besides code reviews
  \item Chapter \ref{chapter:litsurvey} gives the main results of our review, with
  an examination of the merits of reviews in general, the benefits and
  negatives of code reviews and the open questions we have found in reviewing
  the state of the art
  \item Chapter \ref{chapter:experiment} discusses a series of experiments to
  contribute toward answering these open questions
\end{itemize}
