\section{Introduction}

% Here we motivate the problem from scratch. Usually a concrete
% example will do this well. The first paragraph sets the scene
% and states the problem explicitly. It sets up the second paragraph,
% which will state explicitly your hypothesis for solving it.
% 
% Now comes a short, sharp and succinct exposition of your approach.
% For example, ``It thus seems reasonable to conjecture that a
% combination of the X algorithm and the Y algorithm would produce
% superior results to either algorithm on its own. We show that this is
% indeed the case, and prove that the combined algorithm has complexity,
% on the average, that is an order of magnitude better than either the X
% or the Y algorithm.''
% 
% Then there is a bit more motivation, linking the problem to other
% researchers, but not in detail. You also talk in a general way about
% the methodology you have employed to solve the problem.
% 
% The final paragraph should give the conclusions of your
% paper. Something like ``We conclude that the combined X-Y algorithm is
% better suited to robot control problems than either the X algorithm or
% the Y algorithm'' would be appropriate.
