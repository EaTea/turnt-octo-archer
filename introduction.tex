\section{Introduction --- fixed by 28 July 2013}

% Here we motivate the problem from scratch. Usually a concrete
% example will do this well. The first paragraph sets the scene
% and states the problem explicitly. It sets up the second paragraph,
% which will state explicitly your hypothesis for solving it.
% 
% Now comes a short, sharp and succinct exposition of your approach.
% For example, ``It thus seems reasonable to conjecture that a
% combination of the X algorithm and the Y algorithm would produce
% superior results to either algorithm on its own. We show that this is
% indeed the case, and prove that the combined algorithm has complexity,
% on the average, that is an order of magnitude better than either the X
% or the Y algorithm.''
% 
% Then there is a bit more motivation, linking the problem to other
% researchers, but not in detail. You also talk in a general way about
% the methodology you have employed to solve the problem.
% 
% The final paragraph should give the conclusions of your
% paper. Something like ``We conclude that the combined X-Y algorithm is
% better suited to robot control problems than either the X algorithm or
% the Y algorithm'' would be appropriate.

Suppose Alice and Bob are two share traders who fervently watch the stock market. They
are particularly concerned about the fate of company ABC and are waiting for news on whether ABC is
doing well or poorly. In the middle of their stock market vigil, Alice is given a letter that says
ABC has done particularly well. Bob sees her open the letter and read its contents, but he does not
know what the letter is about.\\
\\
If we wanted to formally model Alice and Bob's stock situation, we must consider how we can model
\begin{enumerate}
	\item the facts Alice and Bob know or believe before Alice obtained the letter
	\item the change in belief or knowledge described by Alice's letter and Bob's observations
	\item the updated facts Alice and Bob know or believe after the letter has been opened
\end{enumerate}

We are particularly concerned with items two and  three --- the updated facts Alice and Bob know.
Given that there are facts that we want Alice and Bob to know or believe, we want to construct an
update that will always achieve this information goal. We specify a series of languages to
construct models of information change that realise information goals.\\
\\
In \cite{hales13synthesis} Hales approaches this and shows that such constructions are possible.
We extend his work by providing a language that can be used to generate models, allowing
translation into an automated framework for formal update construction. We show and example
automation of the construction of these formal updates.\\
\\
A formally described update allows us to verify whether the update takes effect as we expect. We can
determine the outcome of an update and if it is correct, as well as reason about the event and the
knowledge situation before and after the update. This level of correctness and security can be
critical in systems such as communications, as well as economics fields like stock marketing and
auctioning. They can be used to reason about agents playing games and find their uses in how we can
update the knowledges of automated agents.

