\chapter{Project Proposal}

In many software development articles, quality is a vague, implicitly defined
concept.
Denning, Buthmann and Soni each discuss quality and the prevention of defects without explicitly
defining what quality {\em is} \cite{Soni:2010:DP, Buthmann:2010:CQ,
	Denning:1992:ESQ:129617.384272}.
What is common to many definitions of ``quality" is the need to minimise the number of defects and to
meet and exceed customer expectations.
I am particularly interested in the first of these characteristics.\\
\\
The minimisation of defects is itself a subjective criteria that relies on the fitness of a design
with respect to reliability, robustness and maintainability.
I will not consider these within my project and will instead focus on tools and techniques that are
(hopefully) largely independent of previous stages of software development and thus as general as
possible.

\section{Primary Goals}

I would like to examine the current state of the art in detecting and reducing defects in code.
This includes, but is not limited to the investigation of
\begin{itemize}
	\item the effectiveness of defect detection techniques like structured walkthroughs and code
	review
	\item whether employing revision control has influenced software development practice, and if so
	how (especially in the open source scene)
	\item how processes like the Personal Software Process (PSP) and Team Software Process (TSP) are
	affecting software development
\end{itemize}

\section{Secondary Goals}

My project will have auxillary, extension goals, which complement but are not essential to its
completion.
I will outline these three goals.
\begin{itemize}
	\item small-scale testing of some processes and tools to discuss their effectiveness
	\item research into the uptake of, and employment of techniques and tools within industry and the
	open source software scene
	\item the construction of, and discussion about the effectiveness of a personal developer process
	will minimise defects on a per-patch basis
\end{itemize}
