%-----------------------------------------------------------------------------%
%Packages%
\documentclass{cshonours}
\usepackage{amsmath, amsfonts, listings, amssymb, mathtools, amsthm} %Mathematical Expressions package
\usepackage{mathtools}
\usepackage[usenames, dvipsnames]{color} %Color naming packages
\usepackage{geometry}
\usepackage{float}
\usepackage{verbatim} %for code
\usepackage[pdftex]{graphics}
\usepackage{hyperref}
\usepackage{cleveref}
\usepackage{tikz}
\usepackage{comment}
\usepackage{fancyhdr}

\usetikzlibrary{arrows,shapes}

%Graphis Extensions
\DeclareGraphicsExtensions{.png, .jpg, .PNG}
\parindent 0pt

% Predefined things such as commands, etc.
\newcommand{\FIXME}{{\bf FIXME}}
% Drawings of frames

\tikzstyle{vertex}=[circle,fill=black!25,minimum size=20pt,inner sep=0pt]
\tikzstyle{selected vertex} = [vertex, fill=red!24]
\tikzstyle{edge} = [draw,thick,->]
\tikzstyle{weight} = [font=\small]

\setlength{\headheight}{15.2pt}
\pagestyle{fancy}

\fancyhf[FC]{\thepage}
\fancyhf[HL]{Edwin Richard Yucheng Tay, 20529864}
\fancyhf[HR]{CITS5502 Research Project}

%-----------------------------------------------------------------------------%
%Document%

\title{Advancing the Advances: Revisiting Software Inspection}
\author{Edwin Richard Yucheng Tay \\
University of Western Australia \\
email: taye03@student.uwa.edu.au }

\date{\today}

\begin{document}

\maketitle

\begin{abstract}
Code reviews are an effective, cheap and well-established software verification method.
Fagan was one of the first to advocate for them, and in his seminal paper in 1986 summarising their
achievements, he strongly advocated that the software engineering industry move in favour of
improving their usage.
The industry has since adopted software inspections as a go-to technique, but many other
verification and defect prevention techniques have also evolved.\\
\\
In such a climate, is the code review still useful?
Why is it still a necessary part of the software engineering defect detection and prevention
toolkit?
And how does it interact with the other, newer pieces of software engineering machinery that have
since evolved since the inception of reviews?
This paper contributes partial answers towards this line of questioning, and attempts to add more
specific and achievable subgoals that the author hopes would assist in advancing the state of
software engineering knowledge and the usefulness of code reviews in the entire defect detection and
prevention process.
\end{abstract}

\pagebreak

\tableofcontents

\section{Introduction}

Why Model?\\
\\
Epstein poses this as a rhetorical question, with an implicit retort that “You
are a modeller”. His claim is that to predict, analyse or understand a real life
situation is to model it. Furthermore, to construct valid models --- models that
are verifiable and “valid” --- is an artifact of the modelling process. Epstein
motivates this idea with respect to social or historical phenomena (such as
    wars, migrations and diseases) \cite{epstein2008model}.\\
\\
We can equally apply models to the software development process.
COCOMO \cite{boehm1984software} and Putnam's SLIM \cite{putnam1978general} are examples of models of the software development process.
They focus on the idea of costing the software development process.
In particular, modelling allows us to evolve strategies for new projects by analysing previous projects.
That is, insights gathered from previously modelled projects can yield more reliable, accurate insights about a new project.\\
\\
Within this paper, we are especially interested in the defect detection and amendment.
We define a defect as
\begin{quote}
  a misunderstanding or mistranslation of an idea into code that results a
  failure experienced by a user of a software system
\end{quote}

It is clear that from a client perspective, defects negatively affect software that clients use. 
Indeed, how timely the detection and fixes of defects occur is an important part of software development.
\cite{gaugingStakeholderPerc} shows that client perception is negatively
influenced by late fixes and lack of timeliness.
It relates back to the ``quality" of the software development service, a soft and hard-to-define
characteristic that governs the outcomes of many projects.
Furthermore, \cite{green2005impacts} shows that there is a need to measure
processes in order to improve quality and the perception of ``goodness" in a
service.\\
\\
This paper aims to answer the question: what is the ``best” way to find and fix defects in a project?
Implicitly, we must also define what the ``best" is.
Our contribution to answering this question is fourfold.
We will
\begin{enumerate}
  \item construct a basic model of the defect detection and reparation process
  \item use previous defect detection data within our model to measure the effectiveness of different defect testing and fixing strategies
  \item analyse the performance of the strategies we have used, as well as challenging the models we have constructed (and implicitly, of the effectiveness of the strategies we have employed)
	\item analyse what our metrics mean, what it means to perform ``well" according to a metric and
answer what the ``best" strategy is
\end{enumerate}

\chapter{What are defects?} \label{chapter:defects}
Defects are common-place occurrences in software development.
It is almost inevitable that software has defects and flaws, which causes software errors and
faults.\\
\\
But what are defects?
How does literature define defects?
And what are some of the problems that arise from seemingly small defects,
and their major after-effects?\\
\\
This chapter is arranged as follows:
\begin{itemize}
	\item Section \ref{sec:defects:defect} discusses what defects are and why they occur
	\item Section \ref{sec:defects:defEffects} explores some of the problems
  that have resulted from seemingly small defects
\end{itemize}

\section{What is a defect?} \label{sec:defects:defect}

Historically, software defects were thought of as ``bugs".
The term was brought to popularity by Grace Hopper, when describing a case of a dead moth being
responsible for a software problem \FIXME.
The idea that software defects were ``bugs" became a well-established phenomenon, and the humourous
idea of software ``bugs" became a reality.\\
\\
Calling software problems ``bugs" is somewhat divorced from the serious nature of a software
problem.
I agree with \FIXME\ in the essay regarding the etymology of ``bugs" and ``defects".
In my opinion, the semantics of a ``defect" over a ``bug" highlight the pressing and wrong nature of the defect.
As a result, I will refer to any software problems that exist in code as ``defects", rather than
``bugs".\\
\\
Defects have been shown to have many evolutions.
Musa et. al \cite{musa1987software} discuss the evolution of defects, from their origins within
code to their effects upon users in a software system.
The evolution stages Musa et. al describe are
\begin{itemize}
	\item a {\em fault} in the code constitutes a mistake in a piece of code as an embodiment of
		knowledge
	\item an {\em error} in system is a bad state that resulted from a fault
	\item a {\em failure} is caused by an {\em error} in a system
\end{itemize}

\FIXME\ needs to focus on faults, and also discuss my own definition\\
\\
\FIXME\ change this to focus on faults...\\
\\
When a failure occurs it is difficult to trace where the original fault that caused it evolved from.
Machine testing on a finished product is simply not an effective debugging process.
Indeed, it can take as much as 100 times more person hours and resources to effectively track down
where the fault came from, according to Boehm et. al \cite{boehm2005foundations}.
This is supported by similar claims made by Soni \cite{soni2006defect} which are somewhat more
modern --- within it, he found that fixing early saved a factor of 3 times more than fixing defects
in the production stage.
Li et. al \cite{li2010transition} report similar results when changing from a plan-based software
lifecycle approach to an iterative Scrum-based approach, in terms of constant defect fixing and a
more efficient software development process.\\
\\
I believe that this trend, outlined by Boehm and supported by further research is a sensible one.
Woodings \cite{terryLecture4220} notes a further analogy about a dragon.
He notes that it is far easier to slay a dragon when it is a baby, than to try and slay the fully
grown dragon.
This metaphor highlights the importance of fixing a defect early.\\
\\
Fixing early requires detecting defects early.
Some schools of thought ascribe to testing early, and testing early and often (see Test Driven Development, in
subsection \ref{subsec:litsurvey:testDD}).
However, some software characteristics are not easily amenable to machine testing --- how do we test
``security" or ``extensibility", as examples?
What happens if there is not enough code to test?
This motivates the need to find techniques that can be deployed within the implementation stages
when the amount of available code for testing is too small to be amenable to machine verification,
or what we need to verify does not lend itself easily to testing.

\section{The effects of defects} \label{sec:litsurvey:defEffects}

To highlight the effects of seemingly simple errors, I will use three examples of real-world
software bugs that caused a major problem in the resultant systems.
In doing so, it will become clear how small problems that are not caught early can be catastrohic,
and furthermore how the smallest differences in how knowledge is encoded into software can
drastically affect system performance.
By exploring the nature of defects, why they occur and how they interplay with representations of
knowledge, we motivate the need for software to be reviewed by humans, and not just machine-tested
or analysed.\\
\\
The first example is related to video games --- namely, the popular video game ``EVE Online".
EVE is a massively multiplayer role-playing game (MMORPG), with a large user base and a virtual
economy \FIXME.
In December 2007, EVE Online was patched with the ``Trinity" patch, which deleted the ``boot.ini"
file inside the EVE Online software directory.
However, this was coded incorrectly, and by assuming that the ``boot.ini" file was being deleted
within EVE Online's directory, the authors of the patch instead deleted the system ``boot.ini" file
which was essential for Windows XP to run \FIXME.\\
\\
This is a small fault in the code --- the author assumed that one file was being deleted and it
happened to have the same name as another file.
That the name of the file in question is the object used to boot the entire operating system
escalated the issue to severely compromise the systems of hundreds, or possibly thousands of
users.\\
\\
The second example involves security, and a small utility called ``Cryptocat".
Cryptocat offers secure instant messaging services that are not tied to services of any sort.
It is popular amongst activist groups for the purposes of communication in a secure manner \FIXME.
Unfortunately, an old patch in Cryptocat's random number generator, which ensured its keys were
secure, instead made all the keys used to encrypt the communications predictable.
The patch was simply an off-by-one error, but it subsequently compromised the integrity of thousands
of users communications \FIXME.\\
\\
The final example revolves around NASA's Mariner 1 --- an ill-fated voyage that ended with the
Mariner 1 being destroyed during operation.
This was caused by a mathematical formula being misread, which resulted in an incorrect
encoding into software.
As a result, the Mariner 1 destroyed itself, simply because the transcriber had not noticed there
was an overbar in one of the equations.
\FIXME\ what's the formula that screwed up?\\
\\
These defects resulted in a fault in the code, and the fault was so small that it is hard to believe
it caused such a huge problem.
These are faults that come from a human attempting to transcribe their knowledge into a software
system.
Some of these bugs might have been amenable to machine testing, but many of them are not easily
tested.
How do you test for randomness?
What if you misread a formula and your test for the formula was wrong, in addition to the code?\\
\\
Verifying that knowledge is transcribed correctly is thus not a problem that is easily automated and
solved by computers.
This motivates the ideas of review, and the need to look beyond testing via a machine.

\chapter{Defect detection and prevention techniques} \label{chap:otherdets}

This paper argues for the relevance of code reviews and software inspection in today's software
industry setting.
To do so, we must ``set the scene" and discuss the current state-of-the-art in defect strategies in
software engineering.
It is thus worthwhile to introduce and survey other defect detection techniques.
We can analyse their relative strengths and weaknesses, and discuss why subtle
errors might escape these techniques.\\
\\
This will not be an in-depth survey of each technique --- we are focusing on code reviews, rather
than these techniques.
Instead, we will speak to each of these with a summary of recent achievements and usage, a
description of what the technique is at a high level, and what the author belives about the
technique and its usage.\\
\\
To address our claims that code reviews have long term benefits, such as defect
prevention properties, we ought to look at different defect prevention
techniques as well.
We will examine three areas of defect detection, and three areas of defect
prevention.
The two defect detection areas are
\begin{itemize}
	\item Test-driven development (Section \ref{sec:otherdets:TDD})
	\item Automated static analysis (Section \ref{sec:otherdets:static})
\end{itemize}
The two defect prevention techniques are
\begin{itemize}
	\item Defensive programming (Section \ref{sec:otherdets:defProg})
	\item Formal methods and model checking (Section \ref{sec:otherdets:modelCheck})
\end{itemize}

\section{Test-driven development} \label{sec:otherdets:TDD}

Continuous development, and the continually refactoring and refining a program according to client
expectations are the basis of Agile programming \FIXME.
One of the branches of Agile programming is Extreme Programming (XP), defined by \FIXME, in which
the ideas of test-driven development evolved \FIXME.\\
\\
Test-driven development states that tests should be written before the program is run, as a means of
specifying a running ``contract" of how the program should work \FIXME.
Some of its variants include the {\em red-green-refactor} methodology and acceptance-test-driven
development.\\
\\
{\em Red-green-refactor} (or RGR), introduced by Shore \FIXME, is a formalisation of the test-driven mantra in
terms of an easy to visualise process.
A developer codes a purposely wrong and small software module, then codes a test that catches the
error (and thus the test is {\em red}).
Once the test catches the error, and thus proves it is working, the developer makes a small change
to the wrong module to correct it, then re-runs the test.
The test should now be passing ({\em green}), and the developer can then {\em refactor} the code to
improve it.
The process repeats itself and slowly makes incremental changes to the system.\\
\\
\FIXME Figure\\
\\
However, this practice is only as effective as the tester.
If the test suite itself is fragile, it is clear that the tests are almost useless, and could lure a
development team into a false sense of security.
Furthermore, RGR only works a unit-test level, without looking beyond small code modules and at the
system as a whole.
Practitioners admit that this is where system testing and acceptance tests are necessary \FIXME.\\
\\
Acceptance-test-driven development, as defined by \FIXME and \FIXME (though they do not appear to be
the originators of the practice) involves a team-based and client-based approach to test-driven
development.
Instead of writing tests for code, the team aims to construct acceptance tests with the client
early, and continually update these tests.\\
\\
I personally do not like this methodology --- acceptance tests are not at all flexible. 
I think it carries the risk of aiming toward a goal (customer vision/expectations) without constant
feedback on whether the goal is appropriate (because it takes so long to satisfy an acceptance test).
This might be because the terminology ``acceptance test" is an artefact of older lifecycle
approaches, and does not translate well to an Agile practice.
One might argue that ``Acceptance Test-Driven Development" is essentially Agile, but this does not
take into account the ideas of ``regression acceptance test suites" (which seem counter-intuitive",
or the expense required to change an acceptance test.\\
\\
Overall, these two methodologies are some differing ideas about test-driven development.
I believe they are essential parts of the software engineering process.
To develop good sosftware, one ought to have code tested continuously, and the tests give the
original developer some measure of confidence in their code.\\
\\
However, it is clear that tests rely upon the confidence of the tester, and the unit tests are only
as good as the person who wrote the software being tested.
Furthermore, the unit tests are an encoding of that developer's understanding of the module's
functionality, and it captures their assumptions and conceptions about the software.
This understanding can easily be flawed or misguided, and that in turn affects the software quality
and the test suite quality until a larger problem (an error or failure) occurs.
Thus, test-driven development (to my mind) cannot find the larger or subtler problems at a code module level, or is not flexible enough to adapt to
finding these large problems.

\section{Automated static analysis} \label{sec:otherdets:static}

Imagine having a spell- and grammar-checker in your document editor, except it analysed your code,
instead.
It detected bad code patterns (code smells \FIXME), potential memory leaks and made optimisations
for you (and it also does spell- and grammar-checking on your documentation).
This is the domain of automated static analysis --- having a machine inspect code and detect the
presence of possible faults \FIXME.\\
\\
Much work has been done to improve the quality of static analysis since Fagan's time.
Biffl \cite{Biffl2012BenefitAutomatedStaticAnalysis} found that small to medium scale enterprises
found benefits in introducing static analysis tools.
There was a unified response from both management and development teams that the static analysis
tools were useful and helpful.
Furthermore, the authors found that it typically took under one hour to
introduce these analysis tools.\\
\\
Interestingly, the authors were able to introduce ``architectural conformance" of a system into
these enterprises.
They could determine circular dependencies or points where the architecture and abstraction layers
were circumvented for ease of code production.
This shows some of the power of static code analysis, in that it has begun to, at a high level,
detect problems in an entire system.\\
\\
Contrastingly, Thung et. al \cite{Thung:2012:EWD:2351676.2351685} find that static analysis still
missed many defects.
They compared the number of false negatives that the static analysis systems had found,
and noted that although they found many defects there were a large range of defects that common
static analysis tools missed.
Their findings suggested that defect detection through static analysis still had a ways to go, with
regards to missing a significant number of defects.\\
\\
Automated static analysis is thus a contender to replace humans in code reviews, since automated
static analysis is essentially doing what a human code reviewer does.
They detect faults in the code by examining it against a set of pre-determined criteria.
These faults can be corrected early before they escalate into major errors and failures.
My own opinion is that such analyses are still rudimentary, and that we should not so quickly
diiscard the critical thinking of a human.\\
\\
Furthermore, Biffl et. al \cite{Biffl2012BenefitAutomatedStaticAnalysis} suggest that there is a need to
select a ``template" for system-wide static analysis.
In my opinion, this further suggests that static analysis is not amenable to large changes in
system-wide software architecture, and could potentially take longer to adapt than a human would
This is simply conjecture and opinion, though, but to my mind it justifies the need for human code
reviewers, rather than just automated static analysis tools.

\section{Defensive programming} \label{sec:otherdets:defProg}

Defensive programming, also known as ``secure programming", looks at eliminating assumptions from
a developer's mindset.
The code developed under defensive programming is centred around accounting for all possibilities,
employing assertions, whitelisting over blacklisting (checking for a correct state and running code,
rather than checking for an incorrect state and throwing an error) and avoiding double negations (e.g. checking that
the result is not incorrect is a double negation).
The entire basis for defensive programming is to prevent defects.
Part of its motivation is to address security concerns and thus decrease the vulnerability of
software components.\\
\\
\FIXME a more comprehensive list about def prog\\
\\
Campbell \cite{campbell1998defensive}, as well as Qie et. al \cite{qie2002defensive}, have looked at
how defensive programming is useful in preventing defects in embedded systems (where testing may
be more difficult or expensive), as well as building security into programs to make them DoS
(denial-of-service) resistant.
However, Roop \FIXME\ argues against defensive programming when employed in an extreme fashion
mainly due to the unmaintainability and readability of the resultant code.\\
\\
I agree with Roop --- I recently constructed a networking protocol for file transfers in an
assignment, and found that it was very difficult to reason about the code I had written.
In some regards, we employed some of the methodologies and ideas behind defensive programming ---
making no assumptions, having a state-based protocol and accounting for all states, and whitelisting
the correct state, rather than blacklisting incorrect states.
However, this lead to hard to read code that I found difficult to debug.
Tracing which part of the protocol was being executed was hard to determine, and although it worked,
much refactoring could have been performed to ensure that it was more readable.\\
\\
Furthermore, defensive programming, though a good set of guidelines to improve the readability and
maintainability of code, places all the onus upon the developer.
Some of the guidelines, such as ``whitelisting over blacklisting", or ``avoiding double negations",
seem easily representable in a static code analysis tool.
However, others, such as ``checking for all possible errors", can only be enforced by the developer
themselves.
Again, as with test-driven development, much responsibility is placed upon the developer to adhere
to these guidelines.
But a developer is a human and is error-prone --- who can enforce the guidelines of defensive
programming, if we cannot trust the developer themselves to do so?

\section{Formal methods and model checking} \label{sec:otherdets:modelCheck})

\begin{quote}
Program testing can be used to show the presence of bugs, but never to show their absence.
\end{quote}

So claimed Dijkstra \FIXME, on the subject of verifying the correctness of a program.
At the time, many scientists argued for proofs of correctness for programs --- Dijkstra claimed that
this would be an expensive and unfeasible task \FIXME.
Indeed, as software has grown and expanded, proving that components work has become an arduous and
logistically complex task.\\
\\
Instead of having people and developers prove the correctness of their code, we instead turn to
machines to do so.
{\em Formal methods} are a way to prove the correctness of code, by a formal, verifiable
specification that can be run against the resultant system.
In doing so, we can employ model checkers to verify whether a system will do exactly as it was
specified, by using an automated prover to show that it does.\\
\\
\FIXME SELinux microkernal paragraph\\
\\
Formal methods are clearly desirable for safety critical programs.
Being able to formally show that a program will do exactly what is expected of it is a desirable and
useful property.
However, it is timely to learn and set up such a programme --- model checking and formal methods are
not commonly used, and are an expert field unto themselves.
Furthermore, how can we show that the specification itself is correct?
That there are no defects within the model checker?\\
\\
High setup costs make it hard to motivate the usage of formal methods for the mainstream software
industry, beyond safety critical software.
I hope to, in this paper, speak to more than just those who need to write safety-critical software,
and instead discuss software engineering at large.

\section{Open questions}

We have examined two defect detection, and two defect prevention techniques, and in each we have
identified some areas where they have shortcomings:
\begin{itemize}
	\item many of them rely on the sole developer to be responsible for defect detection or prevention
	\item they might have high initiation costs and be difficult to set up
	\item it is difficult to see subtleties or verify/prevent the existence of subtle design errors
		using these methods
\end{itemize}

The issues we have highlighted with these defect detection/prevention techniques are ones that I hope
to show code reviews can mitigate in this next chapter.

\chapter{Literature Survey} \label{chapter:litsurvey}

There are many benefits from incorporating software inspections, as well as
some drawbacks.
Some benefits, such as high defect detection rates and immediate improvements to software quality, are
obvious.
Perhaps less obvious and more subtle are the long-term benefits, such as future
defect prevention, that code reviews provide.
This chapter will, through a survey of research address these benefits and discuss the state of the
art in research.\\
\\
This chapter is arranged as follows:
\begin{itemize}
	\item Section \ref{sec:litsurvey:reviews} discusses the merits and ideas behind reviews
	\item Section \ref{sec:litsurvey:codeRev} explores code reviews and the different kinds of code
		reviews
	\item Section \ref{sec:litsurvey:defTechs} lists a wide range of techniques for finding defects
	\item Section \ref{sec:litsurvey:strWeak} analyses the strengths and weaknesses of the different
		code reviews
	\item Section \ref{sec:litsurvey:current} discusses recent research and what the state of the art
		in research is
\end{itemize}

\section{Reviews in management} \label{sec:defects:reviews}

\section{Code reviews} \label{sec:litsurvey:codeRev}

\section{Strengths and weaknesses of code reviews} \label{sec:litsurvey:strWeak}

\section{State of the art} \label{sec:litsurvey:current}

\section{Open questions}

\section{Experiment --- TODO 11 October 2013}

\section{Conclusion}

We have simulated multiple runs of a software defect finding-fixing process.
The recommended strategy that we suggest is to prioritise fixing major bugs,
whilst not allowing the defect queue to go over a certain threshold.\\
\\
We believe that the method we used for simulation was not entirely sound, and
would recommend fixing it to improve the relaibility of samples.
From there, the open questions that remain are
\begin{itemize}
  \item can we address the assumptions we made at the beginning and incorporate
  them into our model?
  \item can we automate the strategy comparison process to evolve and converge
  on a strategy that gives the greatest benefit for a set of metrics?
\end{itemize}


% appendices

\appendix
\section*{Project Proposal}

%\FIXME Make this numbered.


\bibliographystyle{acm}
\bibliography{research_project}

\end{document}
