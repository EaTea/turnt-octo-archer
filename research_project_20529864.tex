%-----------------------------------------------------------------------------%
%Packages%
\documentclass[12pt, a4paper, titlepage]{article}
\usepackage{amsmath, amsfonts, listings, amssymb, mathtools, amsthm} %Mathematical Expressions package
\usepackage{mathtools}
\usepackage[usenames, dvipsnames]{color} %Color naming packages
\usepackage[margin=1.5cm]{geometry}
\usepackage{float}
\usepackage{verbatim} %for code
\usepackage[pdftex]{graphics}
\usepackage{hyperref}
\usepackage{cleveref}
\usepackage{tikz}
\usepackage{comment}
\usepackage[nottoc]{tocbibind}
%\usepackage[square]{natbib}
\usepackage{caption}
\usepackage{subcaption}

\usetikzlibrary{arrows,shapes}

%Graphis Extensions
\DeclareGraphicsExtensions{.png, .jpg}
\parindent 0pt

% Predefined things such as commands, etc.

\newtheorem{defn}{Definition}[subsection]
\newtheorem{thm}{Theorem}[subsection]
\newtheorem{lemma}{Lemma}[subsection]
\newtheorem{corr}{Corrollary}[subsection]
\newtheorem*{remrk}{Remark}

\numberwithin{equation}{section}

\newcommand{\cover}{\bigtriangledown}
\newcommand{\sqex}[1]{[{#1}]}
\newcommand{\anex}[1]{\langle {#1} \rangle}
\newcommand{\lang}{\mathcal{L}}
\newcommand{\langRefine}{\lang_{\forall}}
\newcommand{\langActEx}{\lang_{\otimes}}
\newcommand{\langArbAct}{\lang_{\otimes\forall}}
\newcommand{\langProp}{\lang_0}

\newcommand{\AXK}{{\bf K}}
\newcommand{\AXAML}{${\bf AML_{\AXK}}$}
\newcommand{\AXRML}{${\bf RML_{\AXK}}$}
\newcommand{\AXAAML}{${\bf AAML_{\AXK}}$}

\newcommand{\axP}{{\bf P}}
\newcommand{\axK}{{\bf K}}
\newcommand{\axMP}{{\bf MP}}
\newcommand{\axNecK}{{\bf NecK}}
\newcommand{\axAN}{{\bf AN}}
\newcommand{\axAP}{{\bf AP}}
\newcommand{\axAC}{{\bf AC}}
\newcommand{\axAK}{{\bf AK}}
\newcommand{\axAU}{{\bf AU}}
\newcommand{\axNecA}{{\bf NecA}}
\newcommand{\axR}{{\bf R}}
\newcommand{\axRP}{{\bf RP}}
\newcommand{\axRK}{{\bf RK}}
\newcommand{\axRComm}{{\bf RComm}}
\newcommand{\axRDist}{{\bf RDist}}
\newcommand{\axNecR}{{\bf NecR}}

\newcommand{\kripkeClass}{\mathcal{K}}
\newcommand{\eventClass}{\mathcal{AM}}
\newcommand{\insaneClass}{\eventClass_{IN}}
\newcommand{\publicAnnClass}{\eventClass_{PA}}
\newcommand{\treeClass}{\eventClass_{TR}}
\newcommand{\forestClass}{\eventClass_{FOR}}

\newcommand{\FIXME}{{\bf FIXME}}
% Drawings of frames

\tikzstyle{vertex}=[circle,fill=black!25,minimum size=20pt,inner sep=0pt]
\tikzstyle{selected vertex} = [vertex, fill=red!24]
\tikzstyle{edge} = [draw,thick,->]
\tikzstyle{weight} = [font=\small]

%-----------------------------------------------------------------------------%
%Document%

\title{Languages for epistemic event model synthesis in multi-agent systems}
\author{Edwin Richard Yucheng Tay \\
University of Western Australia \\
email: taye03@student.uwa.edu.au }

\date{\today}

\begin{document}

\maketitle

\begin{abstract}

% A paper should begin with an insightful abstract, written in plain
% English and containing no references. It should state clearly the
% problem under consideration, the motivation for that problem, what you
% have achieved and how, and why your results are interesting.  All this
% in one paragraph too!

\end{abstract}

{\bf Keywords:} List, some, keywords, here.

{\bf CR Classification:} A.1, B.2, C.3.

\pagebreak

\tableofcontents

\section{Introduction}

Why Model?\\
\\
Epstein poses this as a rhetorical question, with an implicit retort that “You
are a modeller”. His claim is that to predict, analyse or understand a real life
situation is to model it. Furthermore, to construct valid models --- models that
are verifiable and “valid” --- is an artifact of the modelling process. Epstein
motivates this idea with respect to social or historical phenomena (such as
    wars, migrations and diseases) \cite{epstein2008model}.\\
\\
We can equally apply models to the software development process.
COCOMO \cite{boehm1984software} and Putnam's SLIM \cite{putnam1978general} are examples of models of the software development process.
They focus on the idea of costing the software development process.
In particular, modelling allows us to evolve strategies for new projects by analysing previous projects.
That is, insights gathered from previously modelled projects can yield more reliable, accurate insights about a new project.\\
\\
Within this paper, we are especially interested in the defect detection and amendment.
We define a defect as
\begin{quote}
  a misunderstanding or mistranslation of an idea into code that results a
  failure experienced by a user of a software system
\end{quote}

It is clear that from a client perspective, defects negatively affect software that clients use. 
Indeed, how timely the detection and fixes of defects occur is an important part of software development.
\cite{gaugingStakeholderPerc} shows that client perception is negatively
influenced by late fixes and lack of timeliness.
It relates back to the ``quality" of the software development service, a soft and hard-to-define
characteristic that governs the outcomes of many projects.
Furthermore, \cite{green2005impacts} shows that there is a need to measure
processes in order to improve quality and the perception of ``goodness" in a
service.\\
\\
This paper aims to answer the question: what is the ``best” way to find and fix defects in a project?
Implicitly, we must also define what the ``best" is.
Our contribution to answering this question is fourfold.
We will
\begin{enumerate}
  \item construct a basic model of the defect detection and reparation process
  \item use previous defect detection data within our model to measure the effectiveness of different defect testing and fixing strategies
  \item analyse the performance of the strategies we have used, as well as challenging the models we have constructed (and implicitly, of the effectiveness of the strategies we have employed)
	\item analyse what our metrics mean, what it means to perform ``well" according to a metric and
answer what the ``best" strategy is
\end{enumerate}

\chapter{Literature Survey} \label{chapter:litsurvey}

There are many benefits from incorporating software inspections, as well as
some drawbacks.
Some benefits, such as high defect detection rates and immediate improvements to software quality, are
obvious.
Perhaps less obvious and more subtle are the long-term benefits, such as future
defect prevention, that code reviews provide.
This chapter will, through a survey of research address these benefits and discuss the state of the
art in research.\\
\\
This chapter is arranged as follows:
\begin{itemize}
	\item Section \ref{sec:litsurvey:reviews} discusses the merits and ideas behind reviews
	\item Section \ref{sec:litsurvey:codeRev} explores code reviews and the different kinds of code
		reviews
	\item Section \ref{sec:litsurvey:defTechs} lists a wide range of techniques for finding defects
	\item Section \ref{sec:litsurvey:strWeak} analyses the strengths and weaknesses of the different
		code reviews
	\item Section \ref{sec:litsurvey:current} discusses recent research and what the state of the art
		in research is
\end{itemize}

\section{Reviews in management} \label{sec:defects:reviews}

\section{Code reviews} \label{sec:litsurvey:codeRev}

\section{Strengths and weaknesses of code reviews} \label{sec:litsurvey:strWeak}

\section{State of the art} \label{sec:litsurvey:current}

\section{Open questions}

\include{./preliminary}
\include{./singleagent}
\include{./multiagent}
\section{Conclusion}

We have simulated multiple runs of a software defect finding-fixing process.
The recommended strategy that we suggest is to prioritise fixing major bugs,
whilst not allowing the defect queue to go over a certain threshold.\\
\\
We believe that the method we used for simulation was not entirely sound, and
would recommend fixing it to improve the relaibility of samples.
From there, the open questions that remain are
\begin{itemize}
  \item can we address the assumptions we made at the beginning and incorporate
  them into our model?
  \item can we automate the strategy comparison process to evolve and converge
  on a strategy that gives the greatest benefit for a set of metrics?
\end{itemize}

\section*{Acknowledgements}

I would like to acknowledge Tim French and James Hales for their valued assistance during this final
year engineering project.

% appendices

\bibliographystyle{ieeetr}
\bibliography{thesis}

\end{document}
